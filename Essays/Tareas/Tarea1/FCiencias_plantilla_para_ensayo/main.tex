\documentclass{article}
\usepackage[margin=1in]{geometry} 
\usepackage{amsmath,amsthm,amssymb,amsfonts, fancyhdr, color, comment, graphicx, environ}
\usepackage{xcolor}
\usepackage{mdframed}
\usepackage[shortlabels]{enumitem}
\usepackage{indentfirst}
\usepackage{hyperref}
\renewcommand{\footrulewidth}{0.8 pt}
\hypersetup{
    colorlinks=true,
    linkcolor=blue,
    filecolor=magenta,      
    urlcolor=blue,
}


\pagestyle{fancy}



\newenvironment{problem}[2][Problem]
    { \begin{mdframed}[backgroundcolor=gray!20] \textbf{#1 #2} \\}
    {  \end{mdframed}}


\newenvironment{solution}{\textbf{Solution}}


\lhead{Fausto David Hernández Jasso}
\rhead{Complejidad Computacional} 
\chead{\textbf{}}
\rfoot{Facultad de Ciencias, UNAM}


\begin{document}
\title{\Large Complejidad Computacional  \\[0.5cm]
        \bf\Large Problemas y Algoritmos}
\author{\large Autor: Fausto David Hernández Jasso\\ \ \\}
\date{\large \today}

\makeatletter
    \begin{titlepage}
        \begin{center}
	   { \includegraphics[width=4cm]{facultad-de-ciencias.jpg}}
	   {\ \\ \ \\}
        \vbox{}\vspace{5cm}
            {\@title }\\[3cm] 
            {\@author}
            {\@date\\}

        \end{center}
    \end{titlepage}
\makeatother

\section{Introducción}
\noindent
Los problemas son tan diferentes entre sí y tienen su origen en distintas variedades, algunos 
problemas son fáciles de resolver como puede ser \textbf{ordenar los monedas por lo grande de su diámetro}
o pueden ser sumamente difíciles como \textbf{tener un buen horario en el semestre}. Del razonamiento anterior,
surge la siguiente pregunta: \textbf{¿Qué hace díficil a un problema?}
\newline 
A lo largo de la primera mitad del siglo pasado, prominentes matemáticos como \textbf{Kurt Gödel, Alan Turing}
y \textbf{Alonzo Church} hicieron un trabajo excepcional al descubrir que ciertos problemas no podían ser 
resultos por las computadoras, una de las consecuencias de éste trabajo es que en particular 
\textbf{Alan Turing} en su artículo \textit{ "On Computable Numbers, with an Application to the Entscheidungsproblem"}
demostró que no era posible determinar sí un premisa matemática es verdadera o falsa \textit{(Alonzo Church demostró 
lo mismo a través del cálculo lambda, en su famoso)} \textbf{Teorema de Church}. Lo anterior implica que no existe 
un algoritmo que pueda realizar ésta tarea. 
\newline 
Así, a través de la computabilidad podemos dar una clasificación a los problemas sí éstos pueden ser resultos por un
módelo de cómputo o no, a diferencia de la complejidad que ésta nos permite clasificar los problemas en sí éstos son 
difíciles o fáciles. 

\section{Desarrollo}
\noindent
La teoría de la complejidad es el enfoque matemático del estudio de las cosas computables \textit{(es decir, aquellas que 
pueden ser calculados a través de un algoritmo ejecutado por algún modelo de cómputo)}, y de los recursos que se requieren 
utilizar para realizar el cómputo. A lo largo de la historia de la humanidad, las personas se han encargado de calcular 
ciertas cosas por su propia mano, ejemplificando, cuando una persona iba a comprar su comida a algún local, a inicios 
del siglo pasado, la persona que le atendía, calculaba el precio total sin ayuda de una computadora, sino simplemente lo 
hacía siguiendo el algoritmo para sumar, sin embargo, ahora ya no sucede así, ya que podemos realizar éste y muchos 
más cálculos con ayuda de una computadora. Pero ¿qué es calculable?, ¿para qué cosas podemos escribir un algoritmo 
que sea ejecutado por una computadora y evitar calcularlo a mano?, éstas preguntas se responden a través de la 
definición de computabilidad, en esencia y no siendo formal estrictamente, algo es computable sí existe un algoritmo 
que puede calcularlo. 
\newline 
En el primer parráfo de éste texto hablamos acerca de los problemas, utilizaremos la siguiente definición para referirnos 
a un problema:
\newline 
\textit{Definimos a un problema \(P\) como el problema de decisión para un lenguaje formal \(L\), esto es, dada una cadena de entrada 
\(s\), determinaremos sí \(s\) pertenece al lenguaje \(L\) o no.}
\newline 
Hacemos la observación que pensar a los problemas como lenguajes formales, está lleno de ventajas, ya que podemos utilizar 
todo el trabajo creado para los \textbf{lenguajes recursivos} y \textbf{recursivamente enumarables} para resolver los problemas, 
desde ahora nuestro estudio estará concentrado en la \textbf{clase compleja} \textit{(complexity class)}, que la definimos 
como el conjunto de lenguajes que pueden ser solucionados bajo una restricción de recursos.
\newline 
¿Cómo saber cuando un problema (lenguaje) es más complejo que otro?, esencialmente definiremos la complejidad 
de un lenguaje tomando en cuenta el recursos gastados para resolver el problema de decisión relacionado con 
dicho lenguaje. 
\newline 
Por lo explicado en el parráfo, daremos la siguiente definición puramente matemática:
\newline 
\textit{La complejidad de un lenguaje \(L\) en relación al uso de recursos que requiere para resolver el problema de decisión de 
\(L\), está dado por la función \(f(n)\), ésta función es la cantidad máxima de recursos usados por \(L\), donde \(n\) es el 
tamaño de la cadena \(s\) de entrada.} Nuestro enfoque será cuando \(f(n)\) va creciendo conforme \(n\) se va haciendo 
terriblemente grande. Debido a ésta definición cualquier \textbf{algoritmo de decisión} creado para determinar si una 
cadena \(s\) pertenece o no a un lenguaje \(L\) tendrá dos cotas, una superior y una inferior, la cuales, naturalmente, 
serán definidas como lo máximo de recursos que vamos a usar y lo que al menos tenemos que usar para resolver el problema 
de decisión, respectivamente. 
\newline
Solamente estamos interesados en cotas polinomiales ya que para entradas bastante grandes, se comportan de manera decente.
\newline 
Los modelos de cómputo que utilizaremos son: máquinas y circuitos. A partir de éste punto, hablaremos solamente de las máquinas como modelo de computo.
Una máquina es una abstracción de la computadora actual, y ésta consistirá en leer una entrada, realizar ciertos cálculos sobre su memoria interna y devolver 
una salida. 
\section{Conclusión}
\noindent
Las medidas de recursos para las máquinas será el tiempo y el espacio. De la medida del tiempo sale una clase extremadamente 
importante como lo es a clase \(P\), que es la clase que contiene al conjunto de lenguajes en los que se 
puede resolver el problema de decisión en tiempo polinomial. De la medida del espacio emanan 3 clases:
\begin{itemize}
    \item \textbf{DSPACE(1)}.
    \item \textbf{DSPACE\(\left(\log{n}\right)\)}
    \item \textbf{PSPACE}.
\end{itemize}
Gracias a lo anterior tenemos una clasificación concreta y precisa de los problemas, ya sea tomando 
como referencia la medida del recurso tiempo o el recurso espacio.
\end{document}