\documentclass{article}
\usepackage[margin=1in]{geometry} 
\usepackage{amsmath,amsthm,amssymb,amsfonts, fancyhdr, color, comment, graphicx, environ}
\usepackage[spanish]{babel}
\usepackage{xcolor}
\usepackage{mdframed}
\usepackage[shortlabels]{enumitem}
\usepackage{indentfirst}
\usepackage{hyperref}
\renewcommand{\footrulewidth}{0.8 pt}
\hypersetup{
    colorlinks=true,
    linkcolor=blue,
    filecolor=magenta,      
    urlcolor=blue,
}


\pagestyle{fancy}



\newenvironment{problem}[2][Problem]
    { \begin{mdframed}[backgroundcolor=gray!20] \textbf{#1 #2} \\}
    {  \end{mdframed}}


\newenvironment{solution}{\textbf{Solution}}


\lhead{Fausto David Hernández Jasso}
\rhead{Complejidad Computacional} 
\chead{\textbf{}}
\rfoot{Facultad de Ciencias, UNAM}

\newtheorem{theorem}{Teorema}[section]
\begin{document}
\title{\Large Complejidad Computacional  \\[0.5cm]
        \bf\Large Problemas y Algoritmos}
\author{\large Autor: Fausto David Hernández Jasso\\ \ \\}
\date{\large \today}

\makeatletter
    \begin{titlepage}
        \begin{center}
	   { \includegraphics[width=4cm]{facultad-de-ciencias.jpg}}
	   {\ \\ \ \\}
        \vbox{}\vspace{5cm}
            {\@title }\\[3cm] 
            {\@author}
            {\@date\\}

        \end{center}
    \end{titlepage}
\makeatother

\section{Relación}
\noindent
\(\mathbf{P} \subseteq \mathbf{NP}\), es decir, todo problema resuelto por un algoritmo determinista en \textbf{tiempo polinomial}, también 
puede ser resulto por un algoritmo no-determinista en \textbf{tiempo polinomial}, ésto se cumple ya que el \textbf{determinismo}
es un caso particular del \textbf{no-determinismo}. La formalización del enunciado anterior es la siguiente:
\newline 
Recordemos que un \textbf{algoritmo no-determinista} cuenta con dos fases:
\begin{itemize}
    \item Fase adivinadora.
    \item Fase verificadora.
\end{itemize}
Entonces sea \(\Pi\) un problema en la clase \(\mathbf{P}\), y sea \(A\) un \textbf{algoritmo determinista} que resuelve a \(\Pi\) en tiempo 
polinomial, entonces construimos un \textbf{algoritmo no-determinista} para \(\Pi\) de la siguiente manera:
\newline 
Utilizamos a \(A\) como la fase verificadora del \textbf{algoritmo no-determinista} e ignoramos la fase adivinadora. Así \(\Pi \in \mathbf{NP}\).
\newline 
A pesar de que es sencillo convertir un \textbf{algoritmo determinista} en un \textbf{algoritmo no-determinista}, lo inverso no lo es,
de hecho no se conoce un método general para convertir un \textbf{algoritmo no-determinista} en tiempo polinomial en un 
\textbf{algoritmo determinista} en tiempo polinomial. De hecho lo único que podemos garantizar es lo siguiente:
\begin{theorem}
    Si \(\Pi \in \mathbf{NP}\), entonces existe un polinomio \(p\) tal que \(\Pi\) puede ser resulto por un \textbf{algoritmo determinista}
    de complejidad \(O\left(2^{p(n)}\right)\)
\end{theorem}
\begin{proof}
    Sea \(\Pi\) un problema tal que \(\Pi\) pertenece a la calse \(\mathbf{NP}\), entonces por definición existe un \textbf{algoritmo no determinista} \(A\)
    que resuelve a \(\Pi\) en \textbf{tiempo polinomial}. Ahora sea \(q(n)\) una cota polinomial superior para la \textbf{complejidad en tiempo} de \(A\).
    Sin pérdida de generalidad asumiremos que \(q(n)\) puede ser evaluado en \textbf{tiempo polinomial} ya que podemos tomar a \(q(n) = c_{1}n^{c_{2}}\) con 
    \(c_{1}, c_{2} \in \mathbb{Z}^{+}\) lo suficientemente grandes para que \(q(n)\) acoten superiormente la \textbf{complejidad en tiempo} de \(A\).
    Recordemos que resolver cualquier problema puede ser visto como un problema de reconocimiento de lenguaje, la transformación del primero al último 
    se hace de la siguiente forma \textit{(no entraremos en muchos detalles)}:
    \begin{itemize}
        \item Codificar el ejemplar del problema en un \textbf{esquema de codificación estándar}.
        \item Determinar sí dicha cadena pertenece al lenguaje dado por el problema.
    \end{itemize}
    Es fácil ver que sí tenemos un algoritmo \(A\) para resolver a \(\Pi\) entonces el problema de \textbf{reconocimiento de lenguaje} asociado 
    a \(\Pi\) se puede diseñar una \textbf{máquina de Turing} que reconozca dicho lenguaje. En éste contexto como \(A\) es un algoritmo no determinista, 
    entonces la \textbf{Máquina de Turing} que reconoce al lenguaje asociado a \(\Pi\) es también \textbf{no-determinista}, recordemos su especificación:
    \begin{itemize}
        \item Un conjunto finito de símbolos \(\varGamma\).
        \item Un conjunto finito de símbolos de entrada \(\Sigma\), tal que \(\Sigma \subset \varGamma\).
        \item Un símbolo especial \(b \in \varGamma\), llamado blanco.
        \item Un conjunto finito de estados \(Q\), un estado inicial \(q_{0}\) y dos estados de paro: \(q_{0}\) y \(q_{N}\).
        \item Una función de transición \(d \ : \ \left(Q \setminus \{q_{0}, q_{N}\} \right) \times \varGamma \longrightarrow Q \times \varGamma \times \{-1, +1\}\)
    \end{itemize}
    Recordemos que la \textbf{Máquina de Turing No-Determinista} utiliza su ``módulo adivinador`` que se encarga 
    de formar cadenas arbitrarias que pertenecen \(\varGamma^{\star}\) y después se hace la verificación sí dicha 
    cadena formada por el ``módulo adivinador`` es aceptada. 
    \newline 
    Por la explicación anterior sabemos que para toda entrada \(i\) de longitud \(n\), entonces debe de existir una
    cadena \(s_{i}\) de longitud a lo más \(q(n)\) formada por el ``módulo adivinador`` que es reconocida por la fase 
    verificadora de \(A\) y ésta verificación no tomará más de \(q(n)\) \textit{(ya que \(q(n)\) es cota de la 
    complejidad en tiempo)}. Por la definición de la \textbf{Máquina de Turing No-Determinista} sabemos que las 
    cadenas formadas por su ``módulo adivinador`` pertenecen \(\varGamma^{\star}\), por lo tanto para una cadena
    de longitud a lo más \(q\left(n\right)\) debemos considerar a lo más \(|\varGamma|^{q\left(n\right)}\), ya que 
    en el peor de los casos tenemos que \(|\varGamma| = n\), las cadenas que tengan longitud menor que \(q\left(n\right)\) 
    trivialmente las podemos hacer de longitud igual a \(q\left(n\right)\), agregándosles el símbolo blanco \(b\). 
    Ahora podemos determinar de manera \textbf{determinista} cuando \(A\) acepta una entrada de longitud \(n\), simplemente 
    aplicamos la fase verificadora de \(A\) hasta que se detenga en los \(q(n)\) pasos 
    o haga \(q(n)\) pasos de cada una de las \(|\varGamma|^{q(n)}\) posibles cadenas. La entrada es aceptada sí en alguna de 
    las cadenas formadas por el módulo ``adivinador`` lleva a un cómputo aceptado dentro de la cota de tiempo en otro caso 
    no es aceptada. Así hemos formado un algoritmo determinista ya que explícitamente hemos considerado todas las cadenas posibles
    y entre ellas vamos probando todas a ver cuál es aceptada sí es que existe alguna que es aceptada. Sin embargo la complejidad 
    de éste algoritmo determinista está dada por \(O\left(q(n) \cdot |\varGamma|^{q(n)}\right)\) ya que cada cadena tiene longitud a lo más 
    \(q(n)\) y existen a los más \(|\varGamma|^{q(n)}\) sin embargo eligiendo un polinomio \(p(n)\) adecuado resulta que
    \(O\left(q(n) \cdot |\varGamma|^{q(n)}\right) \subset O\left(2^{p(n)}\right)\) lo que implica que el 
    \textbf{algoritmo determinista} tenga complejidad \(O\left(2^{p(n)}\right)\).
\end{proof}
\end{document}