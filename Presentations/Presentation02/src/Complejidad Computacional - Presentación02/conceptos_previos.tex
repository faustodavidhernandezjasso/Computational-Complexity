\renewcommand{\sectiontitle}{Conceptos previos}
\section{\sectiontitle}
\customToC{currentsection,hideothersubsections}{}

\renewcommand{\subsectiontitle}{Algoritmo de aproximación}
\subsection{\subsectiontitle}
\begin{frame}{\subsectiontitle}
    \begin{itemize}
        \item Sea \(C_{opt}\) lo que tarda el \textbf{algoritmo óptimo} en encontrar una solución para un problema
        de tamaño \(n\), entonces un \textbf{algoritmo de aproximación} para éste problema tiene una aproximación 
        con proporción dada por \(\varrho\left(n\right)\) sí para cada entrada un algoritmo 
        \textit{(que no es el óptimo)} produce una solución para el problema en tiempo \(C\) tal que:
        \[
            \mathbf{max}\left(\frac{C}{C_{opt}}, \frac{C_{opt}}{C}\right)  
        \]
        tal que el último algoritmo es llamado \textbf{algoritmo} \(\varrho\left(n\right)\) \textbf{aproximado}.
    \end{itemize}
\end{frame}
\renewcommand{\subsectiontitle}{PTAS (Polynomial Time Approximation Scheme)}
\subsection{\subsectiontitle}
\begin{frame}{\subsectiontitle}
    \begin{itemize}
        \item Un \textbf{esquema de aproximación} para un problema de optimización es un conjunto de algoritmos tales que dada 
        una \(\varepsilon > 0\), éste conjunto conjunto contienen \((1 + \varepsilon)\) algoritmos de aproximación 
        \textit{(si es que el problema es de minimización)} y contienen \((1 - \varepsilon)\) algoritmos de aproximación 
        \textit{(si es que el problema es de maximiación)} tal que éstos algoritmos de aproximación son ejecutados en 
        tiempo polinomial en \(n\) cuando \(\varepsilon\) está fijo.
    \end{itemize}
\end{frame}
\renewcommand{\subsectiontitle}{FPTAS (Fully Polynomial Time Approximation Scheme)}
\subsection{\subsectiontitle}
\begin{frame}{\subsectiontitle}
    \begin{itemize}
        \item Un \textbf{esquema de aproximación completo} para un problema de optimización que minimiza es n conjunto de algoritmos
        tales que dada una \(\varepsilon > 0\), contiene \((1 + \varepsilon)\) algoritmos de aproximación 
        tal que éstos algoritmos de aproximación son ejecutados en tiempo polinomial tanto en \(n\) como en \(\frac{1}{\varepsilon}\).
    \end{itemize}
\end{frame}